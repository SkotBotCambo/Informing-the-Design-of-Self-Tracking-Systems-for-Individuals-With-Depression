\section{Introduction}

Due to a lack of mental health professionals available in most areas 
of the world, individuals living with depression see a therapist on 
a sparse basis, if at all. In addition to this, some of the symptoms 
of depression make it even more difficult to get treatment. First, 
when an individual is depressed, they have a tendency to disengage 
from many aspects of social life including meeting with their therapist 
and have difficulty completing therapy related tasks like keeping a 
journal. Second, individuals with depression tend to dwell on 
negative moments in the past and have difficulty recalling positive 
moments in the past. This means that when the patient is finally able 
to meet with their therapist, their recollection of the past few weeks 
can be highly skewed toward negative experiences making diagnosis and 
treatment difficult.

The mHealth community has proposed diary and journaling solutions to help
users track their mood \cite{Ahtinen2013}.  While evidence has shown that users
of journaling and diary systems have found them useful for 
practicing cognitive behavioral therapy (CBT), a common issue
is that usage rates drop over time.  Another approach
has been to build mobile applications which can passively detect and
quantify affective states \cite{Burns2011,Lu2012a}. These systems leverage the growing prevalence and sophistication 
of mobile smartphones to both passively collect behavioral data using 
sensors and system logs as well as actively collect experience sampling 
data related to their mood at a particular moment.  By training a 
machine learning algorithm to correctly predict the actively collected 
data representing the user’s mood using only the passively collected data, 
we can develop a system that can track a user’s mood over time, 
helping them to more objectively self-reflect and accurately present 
their life to a mental health practitioner.  However, early systems, 
such as Mobilyze! (Burns 2011), yielded results that are not significantly better 
than chance.

Our study expands on the work done by Burns et al % cite
by extending the experience sampling survey as 
well as integrating the smartphone data
and experience sampling data into a series of weekly 
face-to-face interviews.  These interviews
help us to understand how observations in the 
form of passively collected mobile phone data
relate to the user's experience of depression.
Additionally, the experience sampling data help 
direct the semi-structured interviews which then
help us to understand the reliability of the 
experience sampling survey design.  Finally,
we present an analysis of the passively
and actively collected data for the feasibility of
of a passive self-tracking system for mood
and other attributes of daily behavior
which may help mental health practitioners
and their patients find the proper 
treatment.