\section{Conclusion and Future Steps}
This work explored ways in which passively collected smartphone data 
might be able to predict the mood state of user's living with depression
and found that most methods of machine learning require too much labeled 
data in order to be effective in this task. Specifically, we explored 
the effectiveness of different algorithms and features in this task.
Finally, we present an analysis of the experience sampling responses
and how the participants felt about their responses
in a weekly interview.  Specifically, how users thought this information
could be useful. These accounts of how the user believes this information to be useful
helps us understand the role that personal informatics can play in 
managing the symptoms of depression.  However, before building
such a system, issues concerning the accuracy of the machine learning methods
used in mood-tracking need to be addressed.  Since users are practicing
self-reflection techniques similar to those in cognitive behavioral therapy
when responding to the experience sampling surveys, we intend to design
the next iteration of the application to more deliberately elicite
self-reflection.

\iffalse