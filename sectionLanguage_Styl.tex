\section{Language, Style and Content}

The written and spoken language of SIGCHI is English. Spelling and
                                punctuation may use any dialect of English (e.g., British, Canadian,
                                US, etc.) provided this is done consistently. Hyphenation is
                                optional. To ensure suitability for an international audience, please
                                pay attention to the following:

\begin{itemize}
                                \item Write in a straightforward style.
                                \item Try to avoid long or complex sentence structures.
                                \item Briefly define or explain all technical terms that may be
                                  unfamiliar to readers.
                                \item Explain all acronyms the first time they are used in your text---e.g.,
                                ``Digital Signal Processing (DSP)''.
                                \item Explain local references (e.g., not everyone knows all city
                                  names in a particular country).
                                \item Explain ``insider'' comments. Ensure that your whole audience
                                  understands any reference whose meaning you do not describe (e.g.,
                                  do not assume that everyone has used a Macintosh or a particular
                                  application).
                                \item Explain colloquial language and puns. Understanding phrases like
                                  ``red herring'' may require a local knowledge of English.  Humor and
                                  irony are difficult to translate.
                                \item Use unambiguous forms for culturally localized concepts, such as
                                  times, dates, currencies and numbers (e.g., ``1-5-97'' or ``5/1/97''
                                  may mean 5 January or 1 May, and ``seven o'clock'' may mean 7:00 am or
                                  19:00).  For currencies, indicate equivalences---e.g., ``Participants
                                  were paid 10,000 lire, or roughly \$5.''
                                \item Be careful with the use of gender-specific pronouns (he, she)
                                  and other gendered words (chairman, manpower, man-months). Use
                                  inclusive language that is gender-neutral (e.g., she or he, they,
                                  s/he, chair, staff, staff-hours,
                                  person-years). See~\cite{Schwartz:1995:GBF} for further advice and
                                  examples regarding gender and other personal attributes.
                                \item If possible, use the full (extended) alphabetic character set
                                  for names of persons, institutions, and places (e.g.,
                                  Gr{\o}nb{\ae}k, Lafreni\'ere, S\'anchez, Universit{\"a}t,
                                  Wei{\ss}enbach, Z{\"u}llighoven, \r{A}rhus, etc.).  These characters
                                  are already included in most versions of Times, Helvetica, and Arial
                                  fonts.
                                \end{itemize}